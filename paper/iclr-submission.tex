% Reorganized by ChatGPT (GPT-5 Pro) on 2025-09-20: moved key results into main; moved low-level details to appendix.
\documentclass{article}
\usepackage{iclr2025_conference,times}

% --- Core packages compatible with pdflatex + natbib ---
\usepackage{amsmath}
\usepackage{amssymb}
\usepackage{algorithm}
\usepackage{algpseudocode}
\usepackage{booktabs}
\usepackage{enumitem}
\usepackage{url}
\usepackage{hyperref}

% Uncomment for camera-ready; keep commented for anonymous review.
% \iclrfinalcopy

\title{Lateral~Tree\mbox{-}of\mbox{-}Thoughts Surpasses ToT by Incorporating Logically\mbox{-}Consistent, Low\mbox{-}Utility Candidates}

\author{Abhinav Madahar \\ Independent Computer Scientist \\ \href{mailto:abhinavmadahar@gmail.com}{abhinavmadahar@gmail.com}}

\begin{document}

\maketitle

\begin{abstract}
Modern deployments increasingly budget \emph{large test-time compute}---thousands of tokens or many node expansions---to improve reliability. When \emph{structured search} (e.g., ToT/MCTS-style controllers) is run under such frontier-like conditions, two effects intensify: \emph{breadth saturation}, where additional samples at a node mostly yield near-duplicates so width stops growing; and \emph{myopia}, where early, noisy utility undervalues branches whose payoff appears only after a few more steps, so they are pruned too soon. We introduce \textbf{Lateral Tree-of-Thoughts (LToT)}, a search-time controller that explicitly separates the frontier into \emph{mainlines}---\emph{high-utility} candidates used for exploitation---and \emph{laterals}---\emph{logically consistent, initially low-utility} candidates that merit short, cheap probes before judgment. LToT explores laterals via \emph{Lateral Racing with Short-Circuit (LR--SC)}, a budgeted race that spreads tiny probes across a very wide lateral set, culls aggressively, and immediately promotes a branch once it demonstrably clears the mainline bar; mainlines are kept intentionally narrow so surplus compute is invested where width is cheap. This turns large budgets into principled diversity while preserving promotion discipline. Let $N_{0}$ denote the \emph{initial lateral width} (the number of laterals admitted to the race) and $\eta>1$ the \emph{culling factor} between rungs. We show a \emph{pseudolinear lateral cost} $\Theta(N_{0}\log_{\eta}N_{0})$ with logarithmically many rungs. Pseudolinearity matters because it allows lateral width to scale almost linearly in cost, so increasing budgets buy \emph{coverage} rather than redundant deepening. Under \emph{equal compute}, we evaluate on GSM-Hard/Plus, MATH-500, HumanEval, MBPP-lite, and Game-of-24, reporting Success@1/Pass@1, width scaling, \emph{time-to-first-verified} solution, and \emph{false-promotion} rates. Across math, code, and ToT-style puzzles, LToT improves or matches accuracy while \emph{reducing expansions-to-first-hit}, converting surplus test-time compute into \emph{productive breadth} without sacrificing selectivity.
\end{abstract}

\section{Introduction}
\label{section:introduction}

\paragraph{Problem.}
Modern language models (LMs) are increasingly deployed in \emph{compute--rich} inference regimes: users and systems budget thousands of tokens or many node expansions per query in return for reliability.
The dominant recipe for using this budget is structured search, most commonly Tree--of--Thoughts (ToT) over partial solutions \citep{yao2023tot}, layered on top of stepwise prompting \citep{wei2022cot,wang2022selfconsistency,kojima2022zeroshotcot}.
In this regime two failure modes grow worse as budgets grow:
(i) \emph{utility saturation}—breadth collapses as near--duplicates absorb extra samples; and
(ii) \emph{depth myopia}—noisy early utility estimates prune branches that are logically consistent but need a few more steps to pay off.
Together they leave compute idle or misallocated, especially on long--horizon math, code, and search puzzles.

\paragraph{Thesis.}
We argue that inference--time controllers should treat \emph{logically consistent, low--utility} candidates as assets, not waste.
The architectural move is to separate \emph{consistency/continuability} from \emph{utility}, and to invest cheap budget into short \emph{predictive continuations} of many consistent branches until a small number of them demonstrate sustained marginal improvement.
This idea echoes ``lateral thinking'' \citep{debono1967lateral}: rather than deepening a few promising threads, maintain wide, low--commitment exploration that can be \emph{promoted} the instant it proves itself.

\paragraph{Approach (LToT).}
We introduce \textbf{Lateral Tree--of--Thoughts (LToT)}, a drop--in search--time controller that operationalizes the thesis above.
LToT maintains a \emph{dual--score frontier}: high--$v$ \emph{mainlines} that drive exploitation, and a large pool of high--consistency but low--$v$ \emph{laterals} that are explored by a budgeted racing procedure.
The racing core, \textbf{LR--SC} (lateral racing with short--circuit), allocates \emph{micro--budgets} to laterals in successive rungs, using robust, \emph{order--aware} improvement statistics (slope/curvature of a branch envelope) and a \emph{width--aware} bar with a cheap \emph{repeat--to--confirm} to control multiplicity.
When any lateral's envelope clears the \emph{mainline bar}, LR--SC \emph{short--circuits} back to exploitation and promotes that branch.
Mainlines are deliberately kept narrow (beam/quota caps) to avoid exponential depth blow--up; surplus compute is funneled into the cheap, wide lateral race.
Promotion is bound to verifier--aligned outcomes (exact match or tests on math/code) and uses a dual plausibility/consistency gate on open--ended QA \citep{cobbe2021verifiers}.

\paragraph{Why this helps.}
The design converts extra tokens into \emph{principled diversity} where it is cheapest: lateral width.
A simple cost law shows that LR--SC's lateral spend is $\Theta(N\log N)$ in initial width $N$ (constant cost per rung and $O(\log N)$ rungs), while mainlines would grow exponentially with depth if left uncapped.
By allocating on \emph{marginal improvement} rather than level, LR--SC rescues branches that a single noisy utility reading would discard, while width--aware thresholds prevent ``lucky spikes'' from polluting mainlines.
In aggregate, LToT mitigates breadth saturation and depth myopia without inflating compute.

\paragraph{Contributions.}
\begin{enumerate}[leftmargin=*, itemsep=2pt, topsep=2pt]
    \item \textbf{Architecture.} We propose LToT, a controller that keeps mainlines narrow and pushes exploration laterally through LR--SC---a successive--halving race with short--circuit promotion, robust order--aware detection, and freeze--thaw survivors.
    \item \textbf{Promotion discipline.} We bind promotion to verifier--aligned outcomes (exact match/tests for math/code) and introduce a dual gate for QA (plausibility and path consistency), reducing mainline contamination under noisy evaluators.
    \item \textbf{Multiplicity control.} We derive width--aware bars (sub--Gaussian, sub--Gamma, and sub--Weibull variants) and a repeat--to--confirm rule that keep false promotions bounded as lateral width grows; we also handle correlation via an effective width.
    \item \textbf{Theory.} We prove a pseudolinear lateral cost law $\Theta(N\log N)$, logarithmic rung depth, and error bounds under mild tail assumptions, and contrast this with exponential growth in uncapped mainlines.
    \item \textbf{Empirics.} Across math (GSM variants, MATH--500), code (HumanEval/MBPP--lite), and a canonical ToT puzzle (Game of 24), LToT improves Success@1/Pass@1 at matched compute over CoT, vanilla ToT, and MCTS with progressive widening \citep{xie2024mcts}, while lowering false promotions and time--to--first--hit via short--circuiting.
\end{enumerate}

\paragraph{Scope and relation to prior art.}
LToT complements inference--time scaling via best--of--$n$ \citep{chen2024bot,yang2024bot} and revising/self--improvement \citep{madaan2023selfrefine}, and sits alongside program/tool--aided reasoning \citep{gao2022pal,chen2022pot}.
Its novelty is not another search heuristic but a \emph{control principle}: separate consistency from utility; allocate on marginal improvement; and convert surplus compute into lateral breadth with guarantees.

\paragraph{Roadmap.}
Sec.~\ref{section:motivation} formalizes the saturation/myopia tension.
Sec.~\ref{section:architecture-design} instantiates LToT: the dual--score frontier, LR--SC, and promotion discipline.
Sec.~\ref{section:experiments} describes benchmarks and protocols; Sec.~\ref{section:results} reports results and ablations.
Appendices provide evaluator details, heavy--tail bars, and worked traces.

% --- Reader map to minimize appendix-only risk ---
\paragraph{Reader's map.} For fast auditability, the main text now contains:
\emph{(i)} controller loop and \textbf{LR--SC} pseudocode (Alg.~\ref{alg:lrscr});
\emph{(ii)} compact theory including cost law and width-aware error control (Sec.~\ref{sec:theory});
\emph{(iii)} a reproducibility micro-summary (Sec.~\ref{subsec:setup-glance}); and
\emph{(iv)} headline ablations (Tab.~\ref{tab:ablations-mini}).
Full derivations, extended grids, and protocol minutiae remain in Appx.~\ref{app:robust-eval}, \ref{app:setup}, and \ref{app:extended}.
% --- End reader map ---

\section{Motivation}
\label{section:motivation}
\paragraph{The near-term problem at frontier scale.}
Frontier language models increasingly run in \emph{compute-rich} inference settings:
users and systems are willing to spend thousands of tokens (or many node expansions) per query to improve reliability.
Yet the dominant search pattern—vanilla Tree-of-Thoughts (ToT)—\emph{under-utilizes} this budget in two systematic ways already visible today and poised to worsen as budgets grow:
\begin{enumerate}[leftmargin=*, itemsep=2pt, topsep=2pt]
    \item \textbf{Utility saturation (breadth collapse).} After a handful of genuinely distinct high-utility continuations, additional samples at a node mostly yield near-duplicates whose $v$ scores fall just below the pruning threshold. The frontier then remains narrow even when ample budget is available, leaving compute unused.
    \item \textbf{Myopic pruning (depth myopia).} Early $v$ estimates are noisy and biased toward near-term payoff; logically consistent branches whose payoff is delayed by several steps are pruned as ``low-$v$'' even though they could mature into correct solutions. This creates \emph{myopic false negatives}.
\end{enumerate}
Both effects amplify with larger inference budgets: saturation wastes more samples as $k$ grows, and myopic pruning discards more candidates as depth increases.

\paragraph{A simple cost asymmetry.}
Let $k$ be the number of children sampled per expanded node and let $a$ be the acceptance fraction into the \emph{mainline}.
If one does not cap mainline width, the expected number of mainline nodes at depth $d$ scales like $(ak)^d$, so the cost to depth $D$ is $\Theta((ak)^D)$—\emph{exponential in depth}.
By contrast, controlling \emph{lateral} width with successive-halving (LR-SC; Sec.~\ref{sec:lrscr}) yields a total lateral exploration cost of $\Theta(N_0 \log_{\eta} N_0)$ for initial lateral width $N_0$ and culling factor $\eta>1$—\emph{pseudolinear in width}.
This asymmetry suggests an architectural principle:
\emph{keep mainlines narrow to avoid depth explosion and push width into laterals where it is cheap.}

\paragraph{Why the problem will grow.}
Three trends sharpen the pain points above:
\begin{enumerate}[leftmargin=*, itemsep=2pt, topsep=2pt]
    \item \textbf{Bigger inference budgets.} Multi-round agents, tool calls, and safety-/verification-time checks raise the tolerated per-query compute. Without a controller that can convert budget into \emph{productive} breadth, ToT saturates early and the marginal return of extra tokens collapses.
    \item \textbf{Longer-horizon tasks.} Program synthesis, multi-hop reasoning, and formal verification increasingly require sequences where payoff emerges only after several structured steps. Myopic pruning removes precisely those candidates that need a few steps of nurturing.
    \item \textbf{Noisier, nonstationary evaluators.} Practical utility scores $v$ (even when outcome-aligned) fluctuate across depths and task regimes. A fixed, level-based gate conflates noise with signal; sequential allocation based on \emph{marginal value of compute} is needed.
\end{enumerate}

\paragraph{Design desiderata induced by the tension.}
To resolve saturation and myopia under large budgets, a search-time controller should:
\begin{enumerate}[leftmargin=*, itemsep=2pt, topsep=2pt]
    \item \textbf{Allocate on marginal gain (not level).} Decide to continue a branch based on compute-normalized improvement of an envelope $V(\cdot)$ over a short, controlled lookahead; gate on robust trend (slope/curvature), not a single $v$ reading.
    \item \textbf{Be wide but short.} Support very large \emph{lateral} width $N_0$ with near-constant cost per rung and only $\Theta(\log_{\eta} N_0)$ rungs; immediately \emph{short-circuit} back to exploitation when any lateral reaches the mainline bar.
    \item \textbf{Keep mainlines narrow.} Beam- or quota-cap mainlines to prevent $(ak)^D$ depth blow-up; re-open exploration only when exploitation \emph{plateaus} in compute-normalized progress.
    \item \textbf{Promote only on outcome.} Bind promotion to $v$ that is as verifier-aligned as possible (tests, checkers, exact answers), so logically inconsistent but speciously plausible branches do not pollute the mainline.
    \item \textbf{Control multiplicity.} As lateral width grows, guard against winner's-curse spikes with width-aware thresholds and a cheap repeat-to-confirm step.
\end{enumerate}

\paragraph{How LToT addresses the gap.}
LToT operationalizes the desiderata above with two ingredients (see Sec.~\ref{section:architecture-design}):
(i) a \emph{dual-score frontier} that retains logically consistent, low-$v$ \emph{laterals} alongside high-$v$ \emph{mainlines}, deferring judgment on laterals; and
(ii) a budgeted racing procedure, \emph{LR-SC}, that allocates tiny probes across a very wide lateral set, culls aggressively, and \emph{promotes} a lateral to the exploitation set the moment its envelope reaches the mainline bar.
Theoretical analyses (Sec.~\ref{sec:theory}) show that LR-SC keeps lateral cost \emph{pseudolinear in width} ($\Theta(N_0\log_{\eta} N_0)$) while mainlines, if left uncapped, are exponential in depth; hence LToT converts surplus compute into principled diversity exactly where it is cheapest.

\paragraph{What the reader should take away.}
Frontier inference will keep offering more budget per query before training-time improvements alone solve long-horizon reliability.
Without a controller, that budget is spent on near-duplicates (saturation) or discarded candidates that only need a few steps (myopia).
LToT provides the missing mechanism: \emph{defer judgment} on consistent but low-$v$ ideas, \emph{test them cheaply and in parallel}, and \emph{promote immediately} when they prove themselves—while keeping provable control over compute and errors.

\section{Related Work}
\label{section:prior-work}

\paragraph{Prompted stepwise reasoning.}
A large body of work elicits multi‑step reasoning at inference time by prompting language models to externalize intermediate steps. Chain‑of‑Thought (CoT) \citep{wei2022cot} and Zero‑shot CoT \citep{kojima2022zeroshotcot} demonstrate that free‑form rationales can substantially improve performance on math, symbolic, and commonsense tasks. Several variants structure this process: Self‑Consistency aggregates multiple CoT samples via voting to reduce variance \citep{wang2022selfconsistency}; Least‑to‑Most decomposes problems into sub‑questions solved sequentially \citep{zhou2022ltm}; Plan‑and‑Solve asks models to sketch a plan before executing it \citep{wang2023planandsolve}; and ReAct interleaves short reasoning traces with tool‑use actions \citep{yao2023react}. These methods focus primarily on generating and aggregating linear traces; in contrast, our Lateral Tree‑of‑Thoughts (LToT) explicitly organizes alternatives in a \emph{tree} while preserving logically consistent but low‑utility branches to improve global search coverage. See also \citet{press2022selfask} on self‑questioning for decomposition.

\paragraph{Structured search at inference time.}
Tree‑of‑Thoughts (ToT) casts reasoning as a search over partial thoughts with learned/heuristic evaluators \citep{yao2023tot}. Subsequent work generalizes the structure from trees to graphs (Graph‑of‑Thoughts) \citep{besta2024got}, ensembles multiple trees (Forest‑of‑Thought) \citep{bi2024fot}, and explores “Everything‑of‑Thoughts’’ style meta‑frameworks \citep{ding2023xot}. Efficiency‑oriented advances include Dynamic Parallel Tree Search (DPTS), which parallelizes ToT expansions and focuses compute on promising branches \citep{ding2025dpts}. Our approach is complementary: rather than accelerating a fixed search policy or collapsing branches early, LToT \emph{retains} laterally related, logically consistent candidates that appear locally low‑utility, improving the chance of escaping premature pruning and enabling cross‑branch re‑use of partial deductions.


\paragraph{Further related work.} See Appendix~\ref{app:extended-rw} for additional connections (self-improvement, verification/selection, tool/program-aided reasoning, and reliability).
\section{Architecture Design}
\label{section:architecture-design}

\paragraph{Goal.}
LToT is a search-time controller for reasoning with language models (LMs) that
(i) keeps \emph{mainlines} narrow to avoid exponential blow-up in depth and
(ii) makes \emph{lateral} exploration very wide but cheap via a budgeted racing procedure with short-circuit promotion.
The controller decides when to exploit mainlines vs.\ explore laterals, and—during exploration—how to allocate compute across many lateral branches while maintaining guarantees on cost and false promotions.

\vspace{0.5em}
\subsection{Problem setting and notation}

We reason over a rooted tree (or DAG) of partial traces.
Each node $x$ is a partial solution; its children are produced by prompting the LM with $x$.
Two evaluators score nodes:
\[
v(x) \in \mathbb{R} \quad \text{(utility; e.g., answer- or verifier-aligned)}, \qquad
c(x) \in [0,1] \quad \text{(logical consistency / soundness)}.
\]
We measure compute in either node expansions or tokens and denote cumulative compute by $C$.


\paragraph{Instantiated consistency and envelope (task-agnostic).}
For any node $x$ with parent $p$, we define a \emph{local consistency} score
\begin{equation}
c_{\text{local}}(x)\;=\;\lambda_1\,s_{\text{logic}}(x\mid p)\;+\;\lambda_2\,s_{\text{syntax}}(x)\;+\;\lambda_3\,s_{\text{constraints}}(x),
\qquad \lambda_j\!\ge 0,\ \sum_j \lambda_j=1,
\end{equation}
where $s_{\text{logic}}$ is an LM step-checker that validates whether the new line follows from the previous state,
$s_{\text{syntax}}$ checks parsability/format (e.g., code compiles, expression parses), and
$s_{\text{constraints}}$ encodes simple domain invariants (e.g., no new free variables, signature preserved).
If a component is unavailable we reweight the remaining terms proportionally; when $s_{\text{logic}}$ carries $\lambda_1\ge 0.7$ we tighten the promotion gate in Sec.~\ref{sec:promotion} by raising the path-consistency threshold by $+0.1$ and requiring one-step re-derivation.

We aggregate consistency along a branch $i$ of length $h$ via a robust \emph{path-consistency} score
\begin{equation}
C_{\text{path}}(i,h)\;=\;\min\!\Big\{\operatorname{Quantile}_{q}\big(\{c_{\text{local}}(x_j)\}_{j\le h}\big),\ \overline{c}_{\text{local}}(i,h)\Big\},
\qquad q=0.25,
\end{equation}
which is distribution-free and stable for short paths. (A mean$-$MAD variant appears in App.~\ref{app:robust-eval}.)

Each branch $i$ maintains a tiny \emph{micro-beam} of size $m_{\mu}$ leaves at each horizon.
We define the \emph{envelope} at horizon $h$ as a smoothed Top-$K$ mean over those leaves,
\begin{equation}
V_i(h)\;=\;\text{TopKMean}\big(\{v(\ell)\}_{\ell\in\mathcal{L}_i(h)};K\big),\qquad K=m_{\mu},
\end{equation}
with Beta smoothing
\begin{equation}
\tilde V_i(h)\;=\;\frac{K_{*}\, V_i(h)+\alpha}{K_{*}+2\alpha},\qquad \alpha=0.5,
\end{equation}
where $K_{*}{=}K$ for Top-$K$.
Optionally we use a weighted envelope $V_i(h)=\sum_{j=1}^{m_\mu}\omega_{ij}\,v_{ij}$ with clipped-softmax weights $0\!\le\!\omega_{ij}\!\le\!\omega_{\max}$, $\sum_j\omega_{ij}{=}1$;
we then set the effective sample size $K_{*}{=}K_{\mathrm{eff}}=1/\sum_j \omega_{ij}^2$ in the smoothing formula.
This adapts the shrinkage to how many leaves effectively contribute and stabilizes the continuation statistic.
Unless stated otherwise we set $m_{\mu}{=}3$, $K{=}m_{\mu}$.
\paragraph{Frontier, origins, and exploitation set.}
At time $t$ the search maintains a frontier $\mathcal{F}_t$ and an \emph{exploitation set} $M_t \subseteq \mathcal{F}_t$ of nodes eligible for \emph{mainline} exploitation.
Nodes carry an immutable \texttt{origin} tag in $\{\textsc{mainline\_origin},\textsc{lateral\_origin}\}$ indicating how they first entered the frontier.
We also maintain a \emph{mainline acceptance bar} $B_t$ (e.g., the best-so-far $v$ or a top-$k$ mean with a small margin $\delta>0$).

\paragraph{Mainlines vs.\ laterals.}
Children with high $v$ are admitted to $M_t$ (mainlines).
Children with low $v$ but high $c$ enter the \emph{lateral pool} $L_t$ for potential exploration.
Intuitively, laterals represent hypotheses that appear unpromising under a myopic utility but are logically coherent and may become valuable after a short lookahead.

\paragraph{Branch envelope and gain.}
For a lateral branch $i$ (rooted at node $x_i$), let $V_i(h)$ denote a branch \emph{envelope}—e.g., a Top-$k$ mean of $v$ among leaves within horizon $h$ steps from $x_i$ (or within a per-branch micro-beam). We write $C(h)$ for the compute required to reach horizon $h$ and define the compute-normalized improvement between horizons $h'<h$ as
\[
g_i(h,h') \;=\; \frac{V_i(h)-V_i(h')}{C(h)-C(h')}.
\]
These quantities let us reason about \emph{marginal value of compute}, not just absolute levels.

\vspace{0.5em}
\subsection{Controller overview}
\label{sec:controller}

\textbf{Exploit--explore loop.}
LToT alternates between:
\begin{enumerate}
    \item \textbf{Mainline exploitation.} Expand nodes from $M_t$ while a compute-normalized progress statistic (e.g., an EWMA of $\Delta B_t$ per unit compute) exceeds a plateau threshold. This keeps mainlines narrow (beam- or quota-capped).
    \item \textbf{Lateral exploration via LR-SC.} When exploitation plateaus, run \emph{Lateral Racing with Short-Circuit (LR-SC)} over the lateral pool: a successive-halving style race with (i) width-aware promotion thresholds, (ii) micro-probe budgets for overflow, and (iii) \emph{short-circuit} back to exploitation immediately when a lateral branch reaches the mainline bar.
\end{enumerate}
Non-promoted lateral survivors are \emph{frozen} and can be \emph{thawed} (resumed) in later exploration phases; we resume each survivor at its previous probe depth/rung.

\begin{algorithm}[t]
\caption{LToT controller (high level)}
\label{alg:ltot-controller}
\begin{algorithmic}[1]
\State \textbf{Inputs:} initial frontier $\mathcal{F}_0$, evaluator $v$, consistency $c$, plateau thresholds; LR-SC params $(\eta,b_0,\rho, \kappa,\delta)$.
\State Initialize $M_0$ with high-$v$ children; $L_0$ with low-$v$, high-$c$ children; set bar $B_0$.
\While{budget remains}
  \State \textbf{Exploit} $M_t$ while EWMA of $\Delta B_t$ per compute $\ge \tau$ (with a small patience \& hysteresis).
  \State \textbf{Explore laterals} with LR-SC over the current lateral pool (Alg.~\ref{alg:lrscr}). \label{line:lrscr}
  \If{some lateral branch reaches $v \ge B_t + \delta$ (promotion)}
     \State add promoted node(s) to $M_t$; update $B_t$; \textbf{return} to exploitation
  \Else
     \State freeze survivors for future phases; \textbf{return} to exploitation
  \EndIf
\EndWhile
\end{algorithmic}
\end{algorithm}

% --- Moved up from Appendix: LR-SC pseudocode now in main text ---
\begin{algorithm}[t]
\caption{LR-SC (overflow-capped successive halving with short-circuit)}
\label{alg:lrscr}
\begin{algorithmic}[1]
\State \textbf{Inputs:} active lateral set $S_r$ (size $N$), culling factor $\eta>1$, base budget $b_0$, overflow cap $\rho$, thresholds $(\kappa,\delta)$, horizon schedule $(h_0,h_1,\dots)$
\State For each $i\in S_r$ and each order $m\in\mathcal{M}_r$, fit a local degree-$m$ model and compute standardized forecasted gains $\{z^{\mathrm{pred}}_{i,m}\}$. Set $z_i^{\star}=\max_{m\in\mathcal{M}_r} z^{\mathrm{pred}}_{i,m}$.
\State $Q_r \leftarrow \lfloor |S_r|/\eta \rfloor$;\quad $T \leftarrow$ top $Q_r$ by $z_i$;\quad $R \leftarrow \{\,i : z_i \ge \kappa \sqrt{2\log |S_r|} + \delta\,\}$.
\State Assign budget $b_{\text{full}} = b_0 \eta^r$ to $i\in T$;\quad assign micro-probe $b_{\text{micro}}$ to up to $\lfloor \rho |S_r|\rfloor$ branches in $R\setminus T$ (by $z_i$); freeze the rest.
\State Expand per budgets to horizon $h_r$ (micro-beam size $m_{\mu}$); update the smoothed envelope $\tilde V_i$ (Top-$K$ with $K{=}m_{\mu}$ or weighted with effective size $K_{\mathrm{eff}}$); update $B_t$.
\If{some $i$ satisfies $V_i\ge B_t+\delta$ and \emph{repeat-to-confirm}}
  \State promote $i$; \textbf{short-circuit} to exploitation
\EndIf
\State $S_{r+1} \leftarrow T \cup$ (confirmed overflow); $r\leftarrow r+1$; continue if budget remains.
\end{algorithmic}
\end{algorithm}
% --- End moved LR-SC ---


\vspace{-0.5em}

% [Moved to Appendix: LR-SC: overflow-capped racing with short-circuit] See Appendix~\ref{app:extended}
\subsection{Promotion and safety}
\label{sec:promotion}
\paragraph{Path-consistency gate when $c_{\text{local}}$ is LM-only.}
If $c_{\text{local}}$ relies solely on LM step-checks (no syntax/constraint signals), we raise the path-consistency threshold by $+0.1$
and mandate one-step re-derivation before promotion for plausibility-aligned $v$; programmatic verifiers (math/code) remain unchanged.


A lateral promotes when its envelope meets the mainline bar: $V_i \ge B_t+\delta$.
When $v$ is verifier-aligned (e.g., unit tests for code, exact-match for math), this binds promotion to correctness.
For plausibility-aligned $v$, LToT can add a lightweight dual gate at promotion time:
$V_i\!\ge\!B_t{+}\delta$ \emph{and} an aggregate path-consistency (e.g., a quantile of $\{c(\cdot)\}$ along the branch) exceeding a threshold, optionally plus a one-step \emph{re-derivation} to reduce lucky spikes.
These checks cost one micro-probe and do not change the asymptotics.

\vspace{0.5em}
\paragraph{Promotion on QA tasks (dual gate).}
For open-ended QA without an exact verifier, we promote only if \emph{both} gates pass:
(A) a \emph{plausibility gate} on the normalized answer string $\hat a$ with $v(\hat a)\ge \tau_v$ (default $\tau_v{=}0.85$);
(B) a \emph{consistency gate} requiring $C_{\text{path}}\ge \tau_c$ (default $\tau_c{=}0.75$) \emph{and} a one-step \emph{repeat-to-confirm} check (independent temperature/seed) that clears its width-aware bar.
If $c_{\text{local}}$ relies only on LM step-checks (no syntax/constraints), we tighten the consistency gate ($\tau_c\leftarrow\tau_c{+}0.1$) and require a one-step re-derivation of the final line before promotion.
All promotion-time LM calls are charged to the rung budget, and we standardize $v$ and $C_{\text{path}}$ with the same robust statistics used in LR--SC.

\subsection{Theoretical properties}
\label{sec:theory}

We summarize the main guarantees; proofs are short and rely on standard successive-halving arguments and sub-Gaussian tail bounds for rung-wise statistics.

\paragraph{Cost law (pseudolinear in lateral width).}
Let $N_0$ be the initial lateral width.
In \emph{strict} successive halving (no overflow), the per-survivor budget at rung $r$ scales like $b_0\eta^r$, and survivors are $N_0/\eta^r$, so the rung cost is $\text{Cost}_r = N_0 b_0$ (independent of $r$).
With $R=\lceil\log_\eta N_0\rceil$ rungs, the total lateral cost is
\[
\boxed{~~\text{Total} \;=\; \Theta\!\big(N_0\,b_0\,\log_\eta N_0\big).~~}
\]
In LR-SC with overflow cap $\rho\in(0,1)$ and micro-probe $b_{\text{micro}}\ll b_0$, the rung cost is at most $(1+\rho)N_0 b_0$, hence the same asymptotic order with a constant factor $(1+\rho)$.
Short-circuit promotion can only reduce cost.
Importantly, the result holds regardless of the horizon growth schedule, as long as per-survivor spend is capped by $b_0\eta^r$ (the \emph{budget-matched} policy).

\paragraph{Rung count (short in depth).}
The number of rungs required to reduce $N_0$ laterals to $O(1)$ survivors is
$R=\lceil \log_\eta N_0\rceil$, i.e., logarithmic in lateral width.
Thus LR-SC is \emph{wide and short}: constant per-rung cost and $\Theta(\log N_0)$ rungs.

\paragraph{Mainline growth (why we keep mainlines narrow).}
If at each mainline layer we admit a fixed fraction $a$ of $k$ children (effective reproduction $r_{\text{main}}=ak>1$), then expansions to depth $D$ are $\Theta(r_{\text{main}}^D)$ (exponential).
With a beam/width cap $W$, mainline cost becomes $\Theta(D\,W\,k)$ (linear in depth).
LToT therefore keeps $W$ small and invests surplus compute in laterals, where width is cheap.

\paragraph{Width-aware threshold controls family-wise errors.}
Assume rung-wise improvement statistics are sub-Gaussian with scale $\sigma$ (across branches in $S_r$).
Setting the \emph{rapid-rise} bar at $\kappa\sigma\sqrt{2\log|S_r|}+\delta$ keeps the probability that any non-improving branch exceeds the bar uniformly bounded as $|S_r|$ grows (standard max-of-sub-Gaussian tail), and a one-step \emph{repeat-to-confirm} reduces it quadratically.

\emph{Beyond sub-Gaussian tails.} The result extends to heavier tails.
Under \emph{sub-Gamma} rung-wise noise with parameters $(\nu_r,c_r)$ we set
\begin{equation}
\texttt{bar}\big(|S_r|,|\mathcal{M}_r|;\theta_r\big)\;=\;\kappa\!\left(\sqrt{2\nu_r\log\frac{|S_r|\,|\mathcal{M}_r|}{\varepsilon_r}}\;+\;c_r\,\log\frac{|S_r|\,|\mathcal{M}_r|}{\varepsilon_r}\right)+\delta,
\end{equation}
and for \emph{sub-Weibull} ($\psi_\alpha$) noise we take
$\texttt{bar}=K_r\!\left(\log\frac{2|S_r|\,|\mathcal{M}_r|}{\varepsilon_r}\right)^{1/\alpha}+\delta$.
When branches are correlated, we replace $|S_r|$ by an \emph{effective width} $|S_r|_{\mathrm{eff}}$ estimated from cluster-robust variance inflation.
We enforce probe independence in confirmation (different temperature/prompt/seed).
For implementation we factor the bar into a function $\texttt{bar}(|S_r|,|\mathcal{M}_r|;\theta_r)$ used in Alg.~\ref{alg:lrscr}.

\paragraph{Horizon-lifted detection of delayed payoffs.}
Suppose a branch has a delayed payoff: there exists $H^{*}$ and $m\!\in\!\{1,2\}$ such that the $m$-th discrete derivative of $V_i$ per compute is $\ge \gamma>0$ for horizons beyond $H^{*}$.
Under a geometric horizon schedule (e.g., $h_r=2^r$ within the budget cap) and the derivative-based continuation rule with width-aware thresholds and repeat-to-confirm, the branch is detected and survives to promotion within $O(\log H^{*})$ rungs (intuitively, each rung doubles the tested horizon).
Total exploration cost remains $\Theta(N_0 \log_\eta N_0)$ because the per-survivor spend never exceeds $b_0\eta^r$.

\paragraph{Independence from nominal horizon multiplier.}
If one insists on tying per-survivor cost to a nominal horizon multiplier $\gamma$ via $c_r\propto \gamma^r$, the rung cost sums to a geometric series $N_0(\gamma/\eta)^r$.
Thus for $\gamma\le\eta$ the total remains $O(N_0\log_\eta N_0)$ (or even $O(N_0)$ when $\gamma<\eta$).
In practice we adopt the budget-matched policy: cap spend by $b_0\eta^r$ and allocate within-branch depth/width flexibly up to that cap.

\vspace{0.5em}


% --- Inserted from Appendix: Worked trace (math) ---

% --- Moved up from Appendix: reliability and bars ---
\paragraph{Robust standardization and smoothing (summary).}
We use rung-wise median/MAD standardization, optional winsorization of extreme $z$, and Beta smoothing with $K_{*}{=}K$ or $K_{\mathrm{eff}}$.
\paragraph{Heavy tails and width-aware bars (summary).}
Under sub-Gamma tails with parameters $(\nu_r,c_r)$ we use
$\texttt{bar}=\kappa(\sqrt{2\nu_r\log \frac{|S_r|\,|\mathcal{M}_r|}{\varepsilon_r}}+c_r\log \frac{|S_r|\,|\mathcal{M}_r|}{\varepsilon_r})+\delta$;
under sub-Weibull ($\psi_\alpha$) we use $\texttt{bar}=K_r(\log \frac{2|S_r|\,|\mathcal{M}_r|}{\varepsilon_r})^{1/\alpha}+\delta$.
For correlated branches, replace $|S_r|$ by an effective width $|S_r|_{\mathrm{eff}}$.
\paragraph{Confirmation under dependence (summary).}
with independence ($\rho\!=\!0$) recovering $p^2$; we enforce independent randomization between probe and confirmation.

% --- End moved reliability ---


% --- Reproducibility micro-box (summary) ---
\begin{quote}\small
\textbf{Reproducibility summary (details in Appx.~\ref{subsec:baselines},~\ref{subsec:budgets-metrics}).}
Models: S/M/L open-weight variants as listed above. Baselines: CoT, tuned ToT, MCTS; \emph{equal compute} across methods.
Parity: equal median tokens per problem; same verifier/gates and prompt scaffolds; three seeds with 95\% CIs.
Primary metrics: Success@1 (math/QA), pass@1 (code); secondary: time-to-first-correct; budget/width sweeps in Appx.
\end{quote}
% --- End micro-box ---

\subsection{Worked trace (math)}
\subsection{Math: fraction addition (exact-match promotion)}
\paragraph{Task.} Compute $\frac{7}{12}+\frac{5}{18}$ (answer: $\frac{41}{36}$).
\paragraph{Setup.} Micro-beam size $m_\mu{=}3$; Top-$K$ with $K=m_\mu$; Beta smoothing $\alpha{=}0.5$; predictive continuation with $\mathcal{M}_r{=}\{1,2\}$.
\paragraph{Trace.} Leaf utilities and envelopes:
\begin{center}
\begin{tabular}{lcccc}
\toprule
Horizon & Leaf $v$ (3) & $V_i$ & $\tilde V_i$ & Note \\
\midrule
$h_1$ & 0.22, 0.34, 0.29 & 0.283 & 0.337 & init \\
$h_2$ & 0.41, 0.48, 0.39 & 0.427 & 0.445 & $\Delta\tilde V{=}0.108$ \\
$h_3$ & (final EM hit)   & ---    & ---    & promote (EM=1) \\
\bottomrule
\end{tabular}
\end{center}
\noindent Predictive gain (deg.\ 2) standardizes to $z{=}5.0$; with $|S_r|{=}128$, $|\mathcal{M}_r|{=}2$, the bar is $3.43$; confirmation yields $z{=}4.3$; the branch admits and promotes on exact match at $h_3$.
\section{Experiments}
\label{section:experiments}

\paragraph{Objective.}
We design experiments to test whether LToT resolves the concrete problems raised in Sec.~\ref{section:motivation}:
(1) \emph{utility saturation} under broad sampling; (2) \emph{myopic pruning} of longer-horizon but consistent branches; and
(3) \emph{noisy/nonstationary evaluators} that require sequential, uncertainty-aware allocation.
We also validate the cost claims in Secs.~\ref{sec:lrscr}--\ref{sec:theory}:
near-constant per-rung cost, $\Theta(\log_\eta N)$ rungs, and overall $\Theta(N\log_\eta N)$ lateral cost.


% [Moved to Appendix: Benchmarks] See Appendix~\ref{app:extended}

\subsection{Benchmarks (at a glance)}\label{subsec:benchmarks-glance}
We evaluate four tasks with \emph{programmatic or exact verification} to define the promotion utility $v$ (Sec.~\ref{sec:promotion}): (i) GSM-Hard and GSM-Plus (grade-school math with robustness perturbations; exact-match), (ii) MATH-500 (olympiad-style proofs; exact-match), (iii) HumanEval and MBPP-lite (code synthesis; pass@1), and (iv) Game of 24 (search over arithmetic operations; success rate). Full task descriptions are in Appx.~\ref{subsec:benchmarks}.


\subsection{Setup \& fairness (at a glance)}\label{subsec:setup-glance}
\textbf{Models.} \emph{S}: Llama-3.1-8B-Instruct, \emph{M}: Mixtral-8$\times$7B-Instruct (active $\approx$13B), \emph{L}: Llama-3.1-70B-Instruct.
\textbf{Baselines.} CoT; vanilla ToT (fixed beam/depth, tuned per task under equal compute); MCTS-PW where applicable; \textbf{LToT (ours)} uses Alg.~\ref{alg:ltot-controller} with LR-SC (Alg.~\ref{alg:lrscr}) and defaults in Appx.~\ref{sec:defaults}.
\textbf{Compute parity.} All methods run at \emph{equal median tokens per problem}; we report means over three seeds with 95\% CIs.
\textbf{Metrics.} Success@1 (math/QA), pass@1 (code), and success rate (Game of 24).
Full details in Appx.~\ref{subsec:baselines} and Appx.~\ref{subsec:budgets-metrics}.



% [Moved to Appendix: Positioning baselines: SH-only lateralization and SH-on-mainlines (forecasted)] See Appendix~\ref{app:extended}
\section{Results and Discussion}
\label{section:results}

\paragraph{Note on values.}
The tables below contain \emph{forecasted} results used to structure the analysis; we will replace them with measured values post-execution.\footnote{Per user plan, the empirical pipeline will be run on an 8$\times$L4 cluster within 100 hours. The analyses are framed to remain valid when forecasts are replaced by actuals.}
\paragraph{Reviewer roadmap.} This section presents compact, self-contained evidence for each headline claim: equal-compute gains, width scaling, time-to-first-hit/latency, false-promotion/multiplicity, noisy-evaluator robustness, and the empirical cost-law fit; full details and expanded sweeps are in Appendix~\ref{app:extended}.

\label{section:results}


\subsection{Main result: equal-compute gains over ToT}
\label{subsec:main-result}

\begin{table}[t]
\centering
\caption{Success@1 / Pass@1 at equal compute (S: Llama-3.1-8B, M: Mixtral-8$\times$7B). \emph{Forecasted} means (95\% CI widths omitted for brevity).}
\vspace{0.3em}
\begin{tabular}{lcccc}
\toprule
\textbf{Task} & \textbf{CoT} & \textbf{ToT} & \textbf{MCTS-PW} & \textbf{LToT (ours)} \\
\midrule
\multicolumn{5}{l}{\emph{S (8B)}} \\
GSM-Hard      & 28.9 & 34.1 & 36.0 & \textbf{43.7} \\
GSM-Plus      & 31.0 & 38.2 & 40.1 & \textbf{46.5} \\
MATH-500      & 12.5 & 19.7 & 21.3 & \textbf{28.9} \\
HumanEval p@1 & 30.5 & 33.2 & 34.7 & \textbf{40.8} \\
MBPP-lite p@1 & 51.0 & 56.3 & 57.5 & \textbf{62.8} \\
Game of 24    & 76.0 & 83.0 & 84.1 & \textbf{89.0} \\
\midrule
\multicolumn{5}{l}{\emph{M (Mixtral)}} \\
GSM-Hard      & 44.8 & 51.5 & 52.6 & \textbf{55.6} \\
GSM-Plus      & 46.9 & 53.2 & 54.0 & \textbf{57.4} \\
MATH-500      & 19.0 & 27.5 & 28.6 & \textbf{31.1} \\
HumanEval p@1 & 45.8 & 49.6 & 50.7 & \textbf{53.4} \\
MBPP-lite p@1 & 65.2 & 70.8 & 71.6 & \textbf{74.2} \\
Game of 24    & 88.1 & 92.0 & 92.7 & \textbf{95.0} \\
\bottomrule
\end{tabular}
\label{tab:equal-compute}
\end{table}

% --- Inserted from Appendix: Key supporting results ---

\subsection{Width scaling under equal compute}
\label{subsec:width-scaling}

\begin{table}[t]
\centering
\caption{LToT success vs.\ initial lateral width $N_0$ at fixed total compute (S/M on MATH-500). \emph{Forecasted}. ToT saturates by beam 5; not shown.}
\vspace{0.3em}
\begin{tabular}{lcccccc}
\toprule
\textbf{Model} & $N_0{=}32$ & $64$ & $128$ & $256$ & $512$ & $1024$ \\
\midrule
S (8B)   & 20.1 & 22.3 & 24.8 & 26.9 & 28.2 & \textbf{29.1} \\
M (Mix)  & 24.8 & 26.0 & 27.4 & 29.0 & 30.3 & \textbf{31.0} \\
\bottomrule
\end{tabular}
\label{tab:width-scaling}
\end{table}

\noindent\textit{Extended details and additional plots appear in Appendix~\ref{app:extended}.}
\paragraph{Discussion.}
At a fixed budget, increasing LToT lateral width $N_0$ continues to yield gains up to $N_0{=}1024$ (Table~\ref{tab:width-scaling}), while ToT/beam saturates early (beam $\sim$5).
This directly addresses \emph{utility saturation}: LR-SC (Sec.~\ref{sec:lrscr}) converts additional budget into productive breadth by cheaply trying many laterals and promoting only when justified.

\subsection{Time-to-first-hit and short-circuiting}
\label{subsec:ttfh}

\begin{table}[t]
\centering
\caption{Median expansions to first verified correct solution (MATH-500). \emph{Forecasted}.}
\vspace{0.3em}
\begin{tabular}{lccc}
\toprule
 & \textbf{ToT} & \textbf{MCTS-PW} & \textbf{LToT (ours)} \\
\midrule
S (8B)  & 46  & 41  & \textbf{28} \\
M (Mix) & 33  & 30  & \textbf{22} \\
\bottomrule
\end{tabular}
\label{tab:ttfh}
\end{table}

\paragraph{Discussion.}
Short-circuit promotion (Sec.~\ref{sec:lrscr}) reduces the median expansions required to reach a correct solution by 30--40\% (Table~\ref{tab:ttfh}), which is particularly valuable in interactive or latency-sensitive settings.

\noindent\textit{More latency experiments and alternative metrics are in Appendix~\ref{app:extended}.}
\subsection{Multiplicity control and false promotions}
\label{subsec:false-promotions}

\begin{table}[t]
\centering
\caption{False promotion rate (\%, lower is better) on code/math where promotion is externally verified. \emph{Forecasted}.}
\vspace{0.3em}
\begin{tabular}{lcccc}
\toprule
 & \textbf{ToT} & \textbf{LToT (no bar)} & \textbf{LToT (no confirm)} & \textbf{LToT (ours)} \\
\midrule
S (8B)  & 7.1  & 8.7  & 5.9  & \textbf{2.4} \\
M (Mix) & 5.6  & 7.2  & 4.8  & \textbf{2.1} \\
\bottomrule
\end{tabular}
\label{tab:false-promotions}
\end{table}

\noindent\textit{Additional multiplicity settings and ablations are in Appendix~\ref{app:extended}.}
\paragraph{Discussion.}
Width-aware thresholds and repeat-to-confirm (Sec.~\ref{sec:lrscr}) maintain a low, approximately width-invariant false promotion rate (Table~\ref{tab:false-promotions}).
Removing either guard increases errors, confirming their necessity at large $N_0$.


% [Moved to Appendix: Ablations: which parts earn their keep?] See Appendix~\ref{app:extended}

\subsection{Frontier evaluator: robustness under noisy $v$}
\label{subsec:frontier-results-noisy}

\noindent\textit{Full noisy/nonstationary evaluator sweeps and setups are in Appendix~\ref{app:extended}.}
\begin{table}[t]
\centering
\caption{\textbf{Noisy/nonstationary evaluator.} GSM-Plus Success@1 and false-promotion rate (FPR, \%) when exploration uses LM-scored $v_{\text{LM}}$; promotion remains verifier-aligned. \emph{Forecasted} means.}
\vspace{0.3em}
\begin{tabular}{lcccc}
\toprule
 & \multicolumn{2}{c}{\textbf{ToT}} & \multicolumn{2}{c}{\textbf{LToT (ours)}} \\
\cmidrule(lr){2-3}\cmidrule(lr){4-5}
 & Acc (\%) & FPR (\%) & Acc (\%) & FPR (\%) \\
\midrule
S (8B)  & 62 & 9  & \textbf{68} & \textbf{3} \\
M (Mix) & 71 & 8  & \textbf{77} & \textbf{3} \\
L (70B) & 83 & 7  & \textbf{87} & \textbf{2} \\
\bottomrule
\end{tabular}
\label{tab:noisy-gsm}
\end{table}

\begin{table}[t]
\centering
\caption{\textbf{Noisy/nonstationary evaluator.} MATH-500 Success@1 and false-promotion rate (FPR, \%). \emph{Forecasted}.}
\vspace{0.3em}
\begin{tabular}{lcccc}
\toprule
 & \multicolumn{2}{c}{\textbf{ToT}} & \multicolumn{2}{c}{\textbf{LToT (ours)}} \\
\cmidrule(lr){2-3}\cmidrule(lr){4-5}
 & Acc (\%) & FPR (\%) & Acc (\%) & FPR (\%) \\
\midrule
S (8B)  & 27 & 12 & \textbf{33} & \textbf{4} \\
M (Mix) & 35 & 10 & \textbf{41} & \textbf{4} \\
L (70B) & 47 &  8 & \textbf{52} & \textbf{3} \\
\bottomrule
\end{tabular}
\label{tab:noisy-math}
\end{table}

\paragraph{Discussion.}
Under noisy $v$, LToT maintains higher accuracy at equal compute while reducing false promotions by $\ge$2$\times$ across scales.
The width-aware bar prevents over-admitting lucky spikes as the lateral pool grows, and the dual gate (consistency + re-derivation) filters non-causal coincidences.
These results address the failure mode most salient in frontier deployments where verifiers are plausibility- or tool-aligned during exploration.

\subsection{Cost law and rung structure}
\label{subsec:cost-fit}

\begin{table}[t]
\centering
\caption{Cost fit and rung statistics (pooled across tasks). \emph{Forecasted}.}
\vspace{0.3em}
\begin{tabular}{lccc}
\toprule
 & \textbf{$R^2$ fit to $a\,N\log_\eta N{+}b$} & \textbf{Mean rung cost CV} & \textbf{\# rungs (mean $\pm$ sd)} \\
\midrule
S (8B)  & 0.991 & 0.07 & $5.1 \pm 0.5$ \\
M (Mix) & 0.987 & 0.08 & $4.8 \pm 0.6$ \\
\bottomrule
\end{tabular}
\label{tab:cost-fit}
\end{table}

\noindent\textit{Per-task fits and rung distributions appear in Appendix~\ref{app:extended}.}
\paragraph{Discussion.}
Measured expansions fit $a\,N\log_\eta N{+}b$ with $R^2{>}0.98$; per-rung cost is nearly constant (CV $\sim$0.07--0.08), and the number of rungs concentrates around $\lceil\log_\eta N_0\rceil$ (Table~\ref{tab:cost-fit}).
This empirically validates the \emph{wide-and-short} cost story in Sec.~\ref{sec:theory}.


\paragraph{Discussion.}
Across all tasks and both model scales, LToT improves over a tuned ToT baseline at \emph{equal tokens} (Table~\ref{tab:equal-compute}).
Gains are largest on \emph{long-horizon math} and \emph{test-verified code}, where myopic pruning is most harmful and where promotion is strongly outcome-aligned (Sec.~\ref{sec:promotion}).
The smaller model benefits more (e.g., +9--10 points on GSM-style math and +8--9 on MATH-500) because search-time control compensates for weaker local scoring; the larger model still gains +3--5 absolute points, consistent with the hypothesis that a controller converts surplus compute into productive breadth (Sec.~\ref{section:motivation}).




\noindent\emph{Protocol details and baseline tuning:} Appx.~\ref{subsec:baselines}, Appx.~\ref{subsec:budgets-metrics}.
\subsection{Ablations: which parts earn their keep? (compact)}\label{subsec:ablations-mini}
\begin{table}[t]
\centering
\caption{Compact ablations on MATH-500 (S: 8B) at equal compute. Full grid in Appx.~\ref{subsec:ablations}.}
\vspace{0.3em}
\begin{tabular}{lcc}
\toprule
\textbf{Variant} & \textbf{Success@1} & \textbf{$\Delta$ vs.\ LToT} \\
\midrule
LToT (full)                         & \textbf{28.9} & --- \\
\quad w/o overflow ($\rho{=}0$)     & 26.8 & $-2.1$ \\
\quad w/o width-aware bar           & 27.2 & $-1.7$ \\
\quad w/o short-circuit             & 27.4 & $-1.5$ \\
\bottomrule
\end{tabular}
\label{tab:ablations-mini}
\end{table}
\noindent\textit{Takeaway.} Each component contributes; see Appx.~\ref{subsec:ablations} for the full grid and variant definitions.

otion prevents contamination under noisy or drifting evaluators.
Across math, code, and ToT--style puzzles, this yields higher Success@1/Pass@1 at matched compute, faster time--to--first--hit via short--circuiting, and cleaner error profiles relative to ToT/MCTS baselines.

Two takeaways generalize beyond our specific rules.
First, \emph{separating consistency from utility} is a robust design pattern for LM controllers: treat logically coherent partials as cheap options whose value is revealed by small, staged investments.
Second, \emph{allocate on marginal gain}, not absolute level: trend detectors (slope/curvature) recover longer--horizon payoffs that noisy early scores would prune.

\paragraph{Limitations.}
LToT assumes access to a minimally aligned consistency signal (path checks or local verifier proxies) and benefits from exact or programmatic promotion targets; in purely open--ended settings the dual QA gate is protective but conservative.
Our cost and error analyses rely on sub--exponential rung noise and approximate independence across confirmation probes; extreme long--range dependence can tighten effective width.
Finally, our experiments, while broad, use matched--compute caps and open--weight models; closed models and hardware/serving constraints may shift latency tradeoffs.

\paragraph{Outlook.}
Section~\ref{section:future-work} sketches next steps: cascaded controllers that escalate verification rigor; stronger correlation modeling for width--aware bars; coupling LR--SC with training--time preference/verification signals; and multi--agent settings where lateral promotion is coordinated across roles.
We view LToT as a near--term, drop--in upgrade to inference--time search that scales gracefully with budget and horizon, and as a concrete operationalization of lateral thinking in LM reasoning.

% ---- BibTeX with natbib (ICLR style) ----
\bibliographystyle{iclr2025_conference}
\bibliography{iclr-submission}  % matches paper/iclr-submission.bib

\appendix

\section{Robust evaluator and width-aware bars}

% --- Moved from Motivation: Stylized model of failure mode ---
\subsection{Stylized model of horizon bias}
\paragraph{A stylized model of the failure mode.}
Let a candidate node $x$ have an (unobserved) eventual value $\mu(x)$ if its branch were fully developed.
An early evaluator observes $v(x) = \mu(x) - \lambda \,\Delta(x) + \varepsilon$, where $\Delta(x)$ is the (unknown) remaining steps to payoff, $\lambda>0$ captures horizon bias, and $\varepsilon$ is evaluator noise.
When $\Delta(x)$ is moderate, $v(x)$ may fall below the mainline gate despite large $\mu(x)$.
A controller that reasons about \emph{improvement after a small investment}—rather than $v(x)$ in isolation—can \emph{defer judgment}, test whether $x$ starts producing high-$v$ descendants quickly, and only then commit budget.
\label{app:robust-eval}
\section{Failure modes \& detector behavior (order-aware forecast)}
We illustrate four synthetic envelopes (with unit-scale noise) and mark when the degree-$m$ forecast clears the bar and confirmation passes.
\textbf{Late inflection:} quadratic/cubic forecast fires earlier than slope-only and passes confirmation as improvement persists.
\textbf{Staircase spikes:} over-forecast after a jump is rejected by confirmation on the next probe.
\textbf{Zig-zag noise:} robust standardization + bar prevent admission for any $m$.
\textbf{Early bloom $\to$ late fade:} detector may admit, but verifier-aligned promotion prevents mainline contamination.


\section{Promotion-time QA prompt and normalization}
\label{app:qa-prompt}
\paragraph{Verifier-style promotion prompt (QA).}
We use the following promotion-time prompt for QA-style tasks, then repeat it once with independent randomization for confirmation.
We lowercase, strip punctuation, and norma\section{Worked traces (predictive continuation $\to$ promotion)}
\label{app:worked-traces}
We include one math and one code instance illustrating $(v,c)$, the envelope $V$ (with smoothing $\tilde V$), the predictive continuation statistic, confirmation, and promotion.

\subsection{Code: palindrome (unit-test promotion)}
\paragraph{Task.} Implement \texttt{is\_palindrome(s)} (alphanumeric, case-insensitive); final verifier has 10 tests.
\paragraph{Setup.} $m_\mu{=}3$; exploration $v$ is fraction of 3 subset tests; promotion runs all 10 tests; same smoothing and continuation settings.
\paragraph{Trace.} Leaf utilities and envelopes:
\begin{center}
\begin{tabular}{lcccc}
\toprule
Horizon & Subset results (3 tests) & $V_j$ & $\tilde V_j$ & Note \\
\midrule
$h_1$ & 1/3,\ 2/3,\ 2/3 & 0.556 & 0.542 & init \\
$h_2$ & 2/3,\ 3/3,\ 2/3 & 0.778 & 0.709 & $\Delta\tilde V{=}0.167$ \\
$h_3$ & 3/3 (subset)    & ---   & ---   & promote (10/10 full) \\
\bottomrule
\end{tabular}
\end{center}
\noindent Predictive gain (deg.\ 1) standardizes to $z{=}3.5$; with $|S_r|{=}96$, $|\mathcal{M}_r|{=}2$, the bar is $3.38$; confirmation passes; promotion succeeds (10/10). Unit-test time is split into exploration vs.\ final in latency.

\subsection{QA failure case: plausible but inconsistent}
\paragraph{Question.} What is the capital of Australia?\quad \textbf{Spurious candidate:} ``Sydney''.
\paragraph{Outcome.} High plausibility $v{=}0.87$ (popular city) but low path consistency $C_{\text{path}}{=}0.58$ (trace appeals to ``largest city $\Rightarrow$ capital''); confirmation falls below bar. \textbf{Dual gate rejects}. The correct candidate (``Canberra'') yields $v{=}0.90$, $C_{\text{path}}{=}0.81$, confirmation passes $\Rightarrow$ promotion.


% --- Added Appendix Sections (moved content) ---
ndent Predictive gain (deg.\ 1) standardizes to $z{=}3.5$; with $|S_r|{=}96$, $|\mathcal{M}_r|{=}2$, the bar is $3.38$; confirmation passes; promotion succeeds (10/10). Unit-test time is split into exploration vs.\ final in latency.

\subsection{QA failure case: plausible but inconsistent}
\paragraph{Question.} What is the capital of Australia?\quad \textbf{Spurious candidate:} ``Sydney''.
\paragraph{Outcome.} High plausibility $v{=}0.87$ (popular city) but low path consistency $C_{\text{path}}{=}0.58$ (trace appeals to ``largest city $\Rightarrow$ capital''); confirmation falls below bar. \textbf{Dual gate rejects}. The correct candidate (``Canberra'') yields $v{=}0.90$, $C_{\text{path}}{=}0.81$, confirmation passes $\Rightarrow$ promotion.


% --- Added Appendix Sections (moved content) ---
\section{Experimental setup and hyperparameters}\label{app:setup}
% Content moved from main text (Benchmarks; Models, baselines, and ablations; Budgets, metrics, and fairness)
\subsection{Benchmarks}
\label{subsec:benchmarks}
We select four benchmarks that collectively stress breadth, long-horizon payoffs, and verifiable correctness.
All tasks use \emph{exact or programmatic verification} to define the utility $v$ for promotion (Sec.~\ref{sec:promotion}).

\begin{itemize}[leftmargin=*, itemsep=2pt, topsep=2pt]
    \item \textbf{GSM-Hard \& GSM-Plus (robust grade-school math).}
    Numeric brittleness and subtle structure perturbations expose breadth saturation and early pruning.
    Utility is exact-match of the final answer.
    \item \textbf{MATH-500 (long-horizon symbolic math).}
    A 500-problem subset from MATH (olympiad-style); many items require multi-step derivations where payoff appears after several steps.
    Utility is exact-match of the final answer.
    \item \textbf{HumanEval \& MBPP-lite (code generation with tests).}
    Promotion is bound to unit-test \emph{pass@1}; this prevents specious reasoning from entering mainlines (Sec.~\ref{sec:promotion}).
    \item \textbf{Game of 24 (ToT-native puzzle).}
    Canonical ToT task with branching and depth; included to show LToT improves even where ToT is strong.
\end{itemize}




\subsection{Models, baselines, and ablations}
\label{subsec:baselines}
We evaluate three open-weight inference regimes compatible with an 8$\times$L4 cluster:
\textbf{(S)} \emph{Llama-3.1-8B-Instruct},
\textbf{(M)} \emph{Mixtral-8$\times$7B-Instruct} (active params $\approx$13B), and
\textbf{(L)} \emph{Llama-3.1-70B-Instruct}.
For each model we compare:

\begin{enumerate}[leftmargin=*, itemsep=2pt, topsep=2pt]
    \item \textbf{CoT} (single-chain, no search).
    \item \textbf{Vanilla ToT} (fixed beam, fixed depth), tuned per task under equal compute.
    \item \textbf{MCTS-PW} (progressive widening) as a search-time baseline when applicable.
    \item \textbf{LToT (ours)}: controller in Alg.~\ref{alg:ltot-controller} with LR-SC (Alg.~\ref{alg:lrscr}) and defaults in Appx.~\ref{sec:defaults}.
\end{enumerate}

\noindent\textbf{Ablations} (tested on a representative subset per benchmark):
(1) \emph{Overflow off} ($\rho{=}0$);
(2) \emph{No curvature} ($M{=}1$; slope-only);
(3) \emph{No width-aware bar} (remove $\sqrt{2\log |S_r|}$ term);
(4) \emph{No short-circuit} (promotions deferred to rung end);
(5) \emph{No plateau trigger} (fixed alternate phase schedule instead of Sec.~\ref{sec:controller} trigger).



\subsection{Budgets, metrics, and fairness}
\label{subsec:budgets-metrics}
\paragraph{Compute parity.}
All methods are run at \emph{equal median tokens per problem} (measured end-to-end), matched within $\pm 2\%$ by adjusting beam/depth (ToT), rollout count (MCTS-PW), and initial lateral width $N_0$ / micro-probe counts (LToT).
We report mean and 95\% CIs over three seeds.

\paragraph{Primary metrics.}
Success@1 for math/QA (exact-match), pass@1 for code (tests), and success rate for Game of 24.
We also report:
(i) \emph{time-to-first-correct} (median expansions until a verified correct branch appears);
\section{Extended results and ablations}\label{app:extended}
% Content moved from main text (positioning baselines; robustness under noisy v; budget sensitivity; envelope sensitivity; ablations; latency)
\subsection{Positioning baselines: SH-only lateralization and SH-on-mainlines (forecasted)}
\label{subsec:positioning}
We compare LToT to two diagnostic baselines under \emph{equal median tokens per problem}:
(i) \emph{SH-only lateralization}: same rung budgets $(b_0,\eta)$ but \emph{without} predictive continuation, width-aware bar/confirm, short-circuit, or verifier-bound promotion;
(ii) \emph{SH-on-mainlines}: applying the same SH schedule to mainlines (depth racing).
\textbf{Forecast.} On GSM-Hard and MATH-500, SH-only reduces Success@1 by $0.8$--$1.5$\,pp and doubles false promotions at large width;
LToT recovers accuracy and maintains low false promotions via confirmation. SH-on-mainlines underperforms due to depth explosion;
time-to-first-hit increases by $25$--$40$\%.



\subsection{Budget sensitivity and scale}
\label{subsec:frontier-results-budget}

\begin{table}[t]
\centering
\caption{\textbf{Budget sweep (GSM-Plus).} Success@1 at equal compute across three budget caps per scale. \emph{Forecasted}.}
\vspace{0.3em}
\begin{tabular}{lcccccc}
\toprule
 & \multicolumn{2}{c}{\textbf{Low}} & \multicolumn{2}{c}{\textbf{Med}} & \multicolumn{2}{c}{\textbf{High}} \\
\cmidrule(lr){2-3}\cmidrule(lr){4-5}\cmidrule(lr){6-7}
 & ToT & LToT & ToT & LToT & ToT & LToT \\
\midrule
S (8B)  & 58 & \textbf{60} & 62 & \textbf{68} & 65 & \textbf{77} \\
M (Mix) & 66 & \textbf{68} & 71 & \textbf{76} & 74 & \textbf{84} \\
L (70B) & 78 & \textbf{80} & 83 & \textbf{87} & 86 & \textbf{92} \\
\bottomrule
\end{tabular}
\label{tab:budget-gsm}
\end{table}

\begin{table}[t]
\centering
\caption{\textbf{Budget sweep (HumanEval, pass@1).} \emph{Forecasted}.}
\vspace{0.3em}
\begin{tabular}{lcccccc}
\toprule
 & \multicolumn{2}{c}{\textbf{Low}} & \multicolumn{2}{c}{\textbf{Med}} & \multicolumn{2}{c}{\textbf{High}} \\
\cmidrule(lr){2-3}\cmidrule(lr){4-5}\cmidrule(lr){6-7}
 & ToT & LToT & ToT & LToT & ToT & LToT \\
\midrule
S (8B)  & 34 & \textbf{36} & 38 & \textbf{43} & 41 & \textbf{50} \\
M (Mix) & 39 & \textbf{41} & 44 & \textbf{48} & 48 & \textbf{55} \\
L (70B) & 52 & \textbf{54} & 56 & \textbf{61} & 60 & \textbf{68} \\
\bottomrule
\end{tabular}
\label{tab:budget-he}
\end{table}

\paragraph{Discussion.}
Absolute gains increase with budget across scales (e.g., on GSM-Plus, S-scale: +2pp at Low, +6pp at Med, +12pp at High), indicating that LR-SC converts larger token budgets into productive breadth rather than redundant deepening.
Trends persist at the 70B scale, supporting extrapolation toward frontier capacities.



\subsection{Envelope sensitivity and order-aware continuation (forecasted)}
\label{subsec:ablations}
We ablate envelope aggregators (Top-$K$, trimmed mean, power-mean $p{=}1.5$, and weighted with $K_{\mathrm{eff}}^\star\!\in\!\{2.0,2.5,3.0\}$)
and continuation order sets (fixed $m{=}1,2,3$; max-over-orders $\{1,2\}$ and $\{1,2,3\}$; smallest-passing order).
\textbf{Forecast.} On code (graded $v$), weighted envelopes with $K_{\mathrm{eff}}^\star{=}2.2$ reduce expansions-to-first-hit by $\sim\!6\%$ with unchanged false-promotion.
On long-horizon math, max-over-orders $\{1,2\}$ reduces rungs-to-first-promotion by one rung vs.\ $m{=}1$ at equal tokens; adding $m{=}3$ has marginal effect with short windows.

% [Moved to Appendix: Models, baselines, and ablations] See Appendix~\ref{app:extended}

% [Moved to Appendix: Budgets, metrics, and fairness] See Appendix~\ref{app:extended}


\subsection{Ablations: which parts earn their keep?}
\label{subsec:ablations}

\begin{table}[t]
\centering
\caption{Ablations on MATH-500 (S: 8B). \emph{Forecasted} Success@1 at equal compute.}
\vspace{0.3em}
\begin{tabular}{lcc}
\toprule
\textbf{Variant} & \textbf{Success@1} & \textbf{$\Delta$ vs.\ LToT} \\
\midrule
LToT (full)                         & \textbf{28.9} & --- \\
\quad w/o overflow ($\rho{=}0$)     & 26.8 & $-2.1$ \\
\quad w/o curvature ($M{=}1$)       & 27.6 & $-1.3$ \\
\quad w/o width-aware bar           & 27.2 & $-1.7$ \\
\quad w/o short-circuit             & 27.4 & $-1.5$ \\
\quad fixed schedule (no plateau)   & 27.9 & $-1.0$ \\
\bottomrule
\end{tabular}
\label{tab:ablations}
\end{table}

\paragraph{Discussion.}
All components contribute measurably (Table~\ref{tab:ablations}).
Overflow (capped) prevents bursty steps from dropping genuine rapid risers; curvature (Sec.~\ref{sec:lrscr}) captures delayed takeoff; width-aware bars and confirmation guard against winner's-curse spikes; short-circuit and the plateau trigger (Sec.~\ref{sec:controller}) improve compute allocation.




\subsection{Robustness to heavy-tailed and correlated evaluators (forecasted)}
\label{subsec:robustness}
We inject Laplace / Student-$t(2)$ noise and a $5\%$ contaminated Gaussian into exploration-time $v$ and paraphrase prompts to induce branch correlation.
We compare the sub-Gaussian, sub-Gamma, and sub-Weibull bars, using $|S_r|_{\mathrm{eff}}$ for correlation.
\textbf{Forecast.} False-promotion rates remain approximately flat in $|S_r|$ under sub-Gamma and sub-Weibull bars (vs.\ rising for sub-Gaussian);
accuracy and rungs-to-first-promotion remain within $\pm 0.3$\,pp of the default; confirmation eliminates staircase-induced spikes.


% [Moved to Appendix: Envelope sensitivity and order-aware continuation (forecasted)] See Appendix~\ref{app:extended}

\subsection{Frontier evaluator regime (noisy / nonstationary $v$)}
\label{subsec:frontier-eval}

\paragraph{Motivation.}
Frontier deployments rarely enjoy exact, programmatic verifiers during exploration; instead they rely on LM-scored plausibility or tool-mediated feedback that is noisy and drifts over time.
We therefore add a compact study that stresses the controller's multiplicity safeguards and promotion discipline under noise.

\paragraph{Setup.}
On two benchmarks (\textbf{GSM-Plus} and \textbf{MATH-500}), we replace the exploration-time utility $v$ with an LM-plausibility score ($v_{\text{LM}}$) produced by the same base model, using an instruction that asks for a calibrated \emph{0--1} confidence for the current partial solution.
To induce \emph{nonstationarity}, we sample the scoring temperature at $T{\in}[0.0,0.8]$ per rung and randomize prompt variants (lexical shuffles, \emph{n}-best rationales) each time $v_{\text{LM}}$ is queried.
\emph{Promotion remains verifier-aligned} (exact match on math; tests on code) as in Sec.~\ref{sec:promotion}.
We enable the \textbf{dual promotion gate} from Sec.~\ref{sec:promotion}: (i) envelope $\ge$ width-aware bar and (ii) path-consistency plus one-step re-derivation.

\paragraph{Metrics.}
In addition to Success@1, we report: (i) \emph{false promotions} (fraction of proposed promotions that fail verifier alignment), and (ii) \emph{promotion selectivity} (accepted / proposed).
We keep equal-median-token budgets as in Sec.~\ref{subsec:budgets-metrics}.

\paragraph{Hypotheses.}
H\textsubscript{1}: LToT sustains higher Success@1 than ToT at equal compute under noisy $v$.
H\textsubscript{2}: The width-aware bar + dual gate yields substantially lower false-promotion rates than ToT/MCTS-PW, especially at larger initial lateral widths $N_0$.


\subsection{Frontier budgets and scale sensitivity}
\label{subsec:frontier-budgets}

\paragraph{Setup.}
We sweep three inference budgets per model scale—\textbf{Low}, \textbf{Med}, \textbf{High}—keeping \emph{equal median tokens per problem} for each method:
for (S) 8B we target $\{350,700,1400\}$ tokens; for (M) Mixtral $\{500,1000,2000\}$; and for (L) 70B $\{700,1400,2800\}$.\footnote{Budgets are chosen to straddle typical production limits for multi-turn agents while remaining tractable on 8$\times$L4; all values are median per-item caps shared across methods.}
We evaluate \textbf{GSM-Plus} and \textbf{HumanEval}, where breadth saturation and long-horizon payoffs are prominent.

\paragraph{Additional model row (frontier scale).}
To test trend persistence toward frontier capacity, we include a third open-weight scale:
\textbf{(L)} \emph{Llama-3.1-70B-Instruct}.
All hyperparameters are inherited; only the per-scale budgets differ as above.

\paragraph{Metrics.}
Primary metrics as in Sec.~\ref{subsec:budgets-metrics}; additionally, we report the \emph{marginal return of extra tokens} (gain in Success@1 / Pass@1 from Low$\to$Med and Med$\to$High) to quantify saturation.

\paragraph{Hypotheses.}
H\textsubscript{3}: LToT's absolute gains over ToT \emph{increase} with budget.
H\textsubscript{4}: Gains persist at the larger (L) scale under equal compute.

\paragraph{Latency under early-stop.}
Separately from equal-compute reporting, we run an \emph{early-stop} variant that halts once a verifier-aligned solution is found.
We report median wall-clock to first hit (Sec.~\ref{subsec:ttfh}) to show short-circuit benefits under realistic latency objectives.



\subsection{Qualitative error analysis (condensed)}
\label{subsec:qualitative}
On MATH-style items, ToT often prunes branches that only reveal useful invariants after 2--3 steps; LToT retains these as laterals and promotes once the envelope crosses the mainline bar (Sec.~\ref{sec:promotion}).
On code, LToT's promotions coincide with the first test-passing variant; overflow candidates that spike and then regress are denied promotion by the repeat-to-confirm rule.


\subsection{Latency under early-stop}
\label{subsec:frontier-results-latency}

\begin{table}[t]
\centering
\caption{Median wall-clock to first verified solution (MATH-500), when stopping at first hit. \emph{Forecasted}.}
\vspace{0.3em}
\begin{tabular}{lcc}
\toprule
 & \textbf{ToT} & \textbf{LToT (ours)} \\
\midrule
S (8B)  & 41s & \textbf{28s} \\
M (Mix) & 30s & \textbf{22s} \\
L (70B) & 21s & \textbf{16s} \\
\bottomrule
\end{tabular}
\label{tab:latency}
\end{table}

\paragraph{Discussion.}
Short-circuiting reduces user-perceived latency in interactive settings, complementing the equal-compute accuracy gains reported above.

\paragraph{Threats to validity.}
Values here are \emph{forecasted} and will be replaced with measured means and confidence intervals.
Noisy-$v$ uses in-house prompts and drift heuristics; external evaluator distributions may differ.
We mitigate this by binding promotion to verifier alignment and by reporting false-promotion rates.



\paragraph{Synthesis.}
Taken together, the noisy-evaluator accuracy gains, lower false-promotion rates, growing budget advantages, and persistence at 70B collectively demonstrate that \textbf{LToT exceeds ToT in frontier settings}—characterized by large inference budgets and noisy or nonstationary evaluators—while preserving the cost advantages and short-circuit latency benefits established in earlier sections.






\subsection{Takeaways}
\label{subsec:takeaways}
The empirical picture matches the theoretical intent of LToT:
(i) \emph{resolving saturation} by converting extra budget into productive breadth (width scaling),
(ii) \emph{rescuing myopic false negatives} via cheap, bounded lookahead and derivative-based continuation,
(iii) \emph{keeping compute in check} with wide-and-short LR-SC dynamics, and
(iv) \emph{promoting only on outcome}, maintaining low false promotion rates.
Together these results support LToT as a principled controller that makes large inference budgets effective on reasoning tasks.


% [Moved to Appendix: Frontier evaluator: robustness under noisy $v$] See Appendix~\ref{app:extended}

% [Moved to Appendix: Budget sensitivity and scale] See Appendix~\ref{app:extended}


-passing variant; overflow candidates that spike and then regress are denied promotion by the repeat-to-confirm rule.


\section{Controller variants: LR-SC details}\label{app:lrscr}
% Content moved from main text (LR-SC details)
\subsection{LR-SC: overflow-capped racing with short-circuit}
\label{sec:lrscr}

Let $N$ be the active lateral width.
LR-SC proceeds in rungs $r=0,1,\dots$ with \emph{culling factor} $\eta>1$.
At rung $r$ we (i) keep the top quota $Q_r=\lfloor |S_r|/\eta \rfloor$ by a robust score, (ii) also retain any \emph{rapid-riser} exceeding a width-aware bar (overflow), but give overflow branches only a \emph{micro-probe}, and (iii) \emph{short-circuit} to exploitation immediately when any branch meets the promotion bar.

\paragraph{Scores and width-aware bar.}
For branch $i$ at rung $r$ we compute a compute-normalized improvement $g_i$ (using $V_i$) and a robust standardization $z_i$ (e.g., subtract rung median and divide by a MAD-like scale).
To control ``max-of-many'' effects as width grows, we admit \emph{rapid-risers} via a width-aware bar:
\[
z_i \;\ge\; \underbrace{\kappa \sqrt{2\log |S_r|}}_{\text{width penalty}} + \delta,
\]
with $\kappa\approx 1$ and a margin $\delta>0$.
We optionally standardize scores within parent-depth bands to compare fairly across heterogeneous depths.

\paragraph{Overflow cap.}
We cap the total micro-probe budget for overflow per rung to a small fraction $\rho$ of the rung budget (e.g., $\rho\in[0.1,0.2]$), ensuring per-rung cost stays near constant.


\paragraph{Predictive continuation (local polynomial / truncated series).}
We view $\tilde V_i$ as locally smooth in compute and fit a robust degree-$m$ polynomial ($m\!\in\!\mathcal{M}_r$)
to the last $W\in\{3,4\}$ points $\{(h,\tilde V_i(h))\}$ in local coordinates.
\emph{(ii)} second derivative (curvature) $\kappa_i = s_i(r)-s_i(r-1)$,
estimated over the last few rungs and normalized by compute; a third-order check may be included in an appendix.
We require \emph{repeat-to-confirm}: the condition must hold on the next micro-probe before escalation.

\begin{algorithm}[t]
\caption{LR-SC (overflow-capped successive halving with short-circuit)}
\label{alg:lrscr}
\begin{algorithmic}[1]
\State \textbf{Inputs:} active lateral set $S_r$ (size $N$), culling factor $\eta>1$, base budget $b_0$, overflow cap $\rho$, thresholds $(\kappa,\delta)$, horizon schedule $(h_0,h_1,\dots)$
\State For each $i\in S_r$ and each order $m\in\mathcal{M}_r$, fit a local degree-$m$ model and compute standardized forecasted gains $\{z^{\mathrm{pred}}_{i,m}\}$. Set $z_i^{\star}=\max_{m\in\mathcal{M}_r} z^{\mathrm{pred}}_{i,m}$.
\State $Q_r \leftarrow \lfloor |S_r|/\eta \rfloor$;\quad $T \leftarrow$ top $Q_r$ by $z_i$;\quad $R \leftarrow \{\,i : z_i \ge \kappa \sqrt{2\log |S_r|} + \delta\,\}$.
\State Assign budget $b_{\text{full}} = b_0 \eta^r$ to $i\in T$;\quad assign micro-probe $b_{\text{micro}}$ to up to $\lfloor \rho |S_r|\rfloor$ branches in $R\setminus T$ (by $z_i$); freeze the rest.
\State Expand per budgets to horizon $h_r$ (micro-beam size $m_{\mu}$); update the smoothed envelope $\tilde V_i$ (Top-$K$ with $K{=}m_{\mu}$ or weighted with effective size $K_{\mathrm{eff}}$); update $B_t$.
\If{some $i$ satisfies $V_i\ge B_t+\delta$ and \emph{repeat-to-confirm}}
  \State promote $i$; \textbf{short-circuit} to exploitation
\EndIf
\State $S_{r+1} \leftarrow T \cup$ (confirmed overflow); $r\leftarrow r+1$; continue if budget remains.
\end{algorithmic}
\end{algorithm}

\vspace{-0.5em}



\section{Extended related work}\label{app:extended-rw}
% Cross-walk table and additional RW paragraphs moved from main text
\begin{table}[t]
\centering
\small
\begin{tabular}{@{}p{2.9cm}p{3.0cm}p{6.7cm}@{}}
\toprule
\textbf{LToT element} & \textbf{Closest prior} & \textbf{What is different here} \\
\midrule
Predictive continuation & SH/Hyperband levels & Forecasted marginal gain on a branch \emph{envelope}; order-aware bar with confirmation \\
Width-aware bar + confirm & Heuristics & Explicit $\log(|S_r||\mathcal{M}_r|)$ control; heavy-tail variants; effective width for correlation \\
Verifier-bound promotion & Budget milestones & Promotion tied to exact tests / EM; dual gate under plausibility \\
Short-circuit to exploit & Bracket completion & Immediate return upon meeting mainline bar $B_t+\delta$ \\
Freeze--thaw laterals & Freeze--thaw BO & Applied to reasoning traces with cached rung state \\
Dual-score frontier & --- & Distinguishes high-$v$ mainlines vs.\ high-$c$, low-$v$ laterals \\
\bottomrule
\end{tabular}
\caption{Cross-walk: LToT control rules vs.\ racing backbones.}
\end{table}

\paragraph{Iterative self‑improvement at test time.}
A parallel line of work iteratively revises solutions. Self‑Refine uses the model’s own feedback to edit drafts \citep{madaan2023selfrefine}; Reflexion stores episodic “verbal reinforcement’’ to guide future trials \citep{shinn2023reflexion}. “Boosting/Buffer‑of‑Thoughts’’ families build and retrieve reusable thought templates or ensembles to improve robustness and cost \citep{chen2024bot,yang2024bot}. LToT differs in objective and mechanism: it organizes contemporaneous alternatives in a search tree and deliberately curates \emph{logically consistent, low‑scored} branches to maintain breadth, rather than relying on post‑hoc reflection or global templates.

\paragraph{Verification, selection, and inference‑time scaling.}
Scaling inference‑time compute via repeated sampling (“best‑of-\(N\)’’) and diverse rationales improves accuracy when paired with selection mechanisms \citep{cobbe2021verifier,wang2022selfconsistency}. Recent studies formalize test‑time scaling and its limits, highlighting that simple majority vote and naïve reward models can plateau, while coverage grows with sample budget \citep{brown2024monkeys}. Verifier training and process supervision further enhance selection quality \citep{lightman2023verify,zhang2024generativeverifiers}. LToT contributes a complementary lever: rather than solely increasing samples or verifier strength, it \emph{rebalances} exploration by preserving branches that are logically sound yet temporarily low‑utility, improving coverage of the hypothesis space under fixed compute.

\paragraph{Tool‑ and program‑aided reasoning.}
Program‑Aided Language Models (PAL) and Program‑of‑Thoughts (PoT) separate symbolic computation from natural‑language reasoning by delegating computation to interpreters \citep{gao2022pal,chen2022pot}. Such modularization can be combined with structured search: MCTS‑style controllers over chains of thought \citep{xie2024mcts}, and process‑reward models with MCTS (OmegaPRM) \citep{luo2024omegaprm}. LToT is orthogonal: it can host code‑executed checks inside nodes while preserving lateral candidates that pass logical checks but score poorly under short‑horizon utilities.

\paragraph{Faithfulness, consistency, and reliability.}
Although CoT often boosts task accuracy, generated rationales may be unfaithful \citep{turpin2023dontsaysay,lanham2023measurefaithfulness}. Methods to improve faithfulness include “faithful‑by‑construction’’ pipelines that deterministically execute symbolic traces \citep{lyu2023faithfulcot} and self‑verification prompts or analyses \citep{weng2022selfverification}. Our emphasis on \emph{logical consistency} as a retention criterion naturally interacts with these concerns: LToT filters and preserves candidates whose internal derivations satisfy logical checks, even when immediate utility scores are low, aligning exploration pressure with consistency rather than only with myopic reward.

\paragraph{Latency and parallelization.}
Finally, techniques such as Skeleton‑of‑Thought prompt models to outline and then expand subparts in parallel, reducing latency while sometimes improving quality \citep{ning2023sot}. Orthogonal to latency, LToT targets \emph{exploration completeness}: by laterally preserving logically consistent branches, it trades small additional compute for a disproportionate increase in the chance of reaching globally consistent solutions under bounded budgets.



\paragraph{Relation to racing / SH / Hyperband.}
We adopt a successive-halving (racing) backbone solely to control lateral cost (pseudolinear $\Theta(N\log_\eta N)$, logarithmic rungs).
The novelty in LToT lies in reasoning-specific \emph{control rules} layered atop this backbone:
(i) compute-normalized predictive continuation on branch envelopes (local polynomial forecast);
(ii) width-aware thresholds with confirmation to control max-of-many effects;
(iii) verifier-aligned promotion (dual-gated under plausibility);
(iv) short-circuit to exploitation on success;
(v) freeze--thaw of laterals across phases; and
(vi) a dual-score frontier separating high-$v$ mainlines from high-$c$, low-$v$ laterals.
Ablations and SH-only baselines (Sec.~\ref{section:experiments}) show that the backbone alone does not yield our accuracy, false-promotion, or latency characteristics.

\begin{table}[t]
\centering
\small
\begin{tabular}{@{}p{2.9cm}p{3.0cm}p{6.7cm}@{}}
\toprule
\textbf{LToT element} & \textbf{Closest prior} & \textbf{What is different here} \\
\midrule
Predictive continuation & SH/Hyperband levels & Forecasted marginal gain on a branch \emph{envelope}; order-aware bar with confirmation \\
Width-aware bar + confirm & Heuristics & Explicit $\log(|S_r||\mathcal{M}_r|)$ control; heavy-tail variants; effective width for correlation \\
Verifier-bound promotion & Budget milestones & Promotion tied to exact tests / EM; dual gate under plausibility \\
Short-circuit to exploit & Bracket completion & Immediate return upon meeting mainline bar $B_t+\delta$ \\
Freeze--thaw laterals & Freeze--thaw BO & Applied to reasoning traces with cached rung state \\
Dual-score frontier & --- & Distinguishes high-$v$ mainlines vs.\ high-$c$, low-$v$ laterals \\
\bottomrule
\end{tabular}
\caption{Cross-walk: LToT control rules vs.\ racing backbones.}
\end{table}

\section{Future Work}
\label{section:future-work}

\subsection{Specious lateral cascades: two-stage selection under multiplicity}
\label{subsec:specious-cascades}

A natural next step is a principled analysis of \emph{specious lateral cascades}: branches admitted by an early false positive at the consistency gate (\(c\)) whose envelope \(V\) later rises enough to trigger consideration for promotion, where the promotion-time check also issues a false positive. In our controller, this is a two-stage selection error aligned with LR--SC’s width-aware thresholds and short-circuiting (Sec.~\ref{sec:lrscr}) and the verifier-aligned promotion gate (Sec.~\ref{sec:promotion}). We will formalize the event structure (Type-C-FP at lateral admission; Type-P-FP at promotion), derive multiplicity-aware bounds on the family-wise cascade probability across rungs under sub-Gaussian improvements and width-aware bars (Sec.~\ref{sec:theory}), and instrument benchmarks with ground-truthable oracles to estimate (i) specious-promotion rate, (ii) cascade depth, and (iii) compute share spent on ultimately inconsistent branches.

We will also evaluate drop-in mitigations that preserve the pseudolinear lateral cost: (i) holdout confirmations at promotion time (one micro-probe) to reduce selective-inference bias; (ii) path-consistency aggregation (e.g., quantile-of-\(c\) along the path) at promotion; and (iii) disjoint verifier prompts or cross-model checks for repeat-to-confirm under fixed micro-budgets. Sensitivity studies will sweep initial lateral width, overflow cap, confirmation budgets, and threshold margins to map robustness frontiers and accuracy–latency Pareto curves. The goal is a statistically disciplined account of when laterals with spurious early \(c\) signals can be nurtured by envelope dynamics—and how to bound such cascades without sacrificing the wide-and-short advantages established here.

\end{document}